\documentclass{article}
\usepackage[utf8]{inputenc}
\usepackage{graphicx}
\usepackage[simplified]{pgf-umlcd}
\usepackage{tikz}
\usepackage{multirow}
\usepackage{float}
\usetikzlibrary{positioning,fit,calc,arrows.meta, shapes}
\usepackage{wrapfig}
\usepackage{listings}
\usepackage{hyperref}
\usepackage{amsmath}
\graphicspath{ {images/} }

%Tot això hauria d'anar en un pkg, però no sé com és fa
\newcommand*{\assignatura}[1]{\gdef\1assignatura{#1}}
\newcommand*{\grup}[1]{\gdef\3grup{#1}}
\newcommand*{\professorat}[1]{\gdef\4professorat{#1}}
\renewcommand{\tablename}{Taula}
\renewcommand{\title}[1]{\gdef\5title{#1}}
\renewcommand{\author}[1]{\gdef\6author{#1}}
\renewcommand{\date}[1]{\gdef\7date{#1}}
\renewcommand{\contentsname}{Índex}
\renewcommand{\listfigurename}{Llista d'imatges}
\renewcommand{\listtablename}{Llista de Taules}
\renewcommand{\maketitle}{ %fa el maketitle de nou
    \begin{titlepage}
        \raggedright{UNIVERSITAT DE LLEIDA \\
            Escola Politècnica Superior \\
            Grau en Enginyeria Informàtica\\
            \1assignatura\\}
            \vspace{5cm}
            \centering\huge{\5title \\}
            \vspace{3cm}
            \large{\6author} \\
            \normalsize{\3grup}
            \vfill
            Professorat : \4professorat \\
            Data : \7date
\end{titlepage}}
%Emplenar a partir d'aquí per a fer el títol : no se com es fa el package
%S'han de renombrar totes, inclús date, si un camp es deixa en blanc no apareix

\tikzset{
	%Style of nodes. Si poses aquí un estil es pot reutilitzar més facilment
	base/.style = {circle, draw=black,
      minimum width=0.75cm, font=\ttfamily,
      text centered},
    dots/.style = {circle, draw=white,
      minimum width=0.75cm, font=\ttfamily,
      text centered},
    last/.style = {base, fill=orange!15},
    remove/.style = {base, fill=red!15},
    change/.style = {base, fill=green!15},Xarxes
    tree/.style = {base, rectangle, minimum height=0.75cm},
    stack/.style = {rectangle, font=\ttfamily, rounded corners, draw=black,
      minimum width=4cm, minimum height=1cm,
      text centered},
   	even/.style = {stack, fill=green!30},
   	odd/.style = {stack, fill=orange!15},
   	blank/.style = {stack, minimum height=0.5cm, draw=white},
   	typetag/.style={rectangle, draw=black!50, font=\ttfamily, anchor=west}
}
\renewcommand{\figurename}{Figura}
\title{Anàlisi de la xarxa mitjançant l'analitzador de protocols de xarxa Wireshark}
\author{Sergi Simón Balcells\\21040111X}
\date{Diumenge 19 de Maig}
\assignatura{XARXES}
\professorat{E. Guitart, C. Mateu}
\grup{GM3}

%Comença el document

\begin{document}
\maketitle
\thispagestyle{empty}

\newpage
\pagenumbering{roman}
\tableofcontents
\listoffigures
\listoftables
\newpage
\pagenumbering{arabic}
\section{Introducció}
\section{Caractarístiques de la xarxa}
% Anàlisi paquest agafar característiques
\subsection{Tipus d'adreçament a la capa de xarxa}
Per a trobar el tipus d'adreçament a la xarxa, s'ha mirat els paquets
tipus ARP per a observar diferents direccions IP de la xarxa.\\
\\
Observant les diferents direccions que es mouen dins de la xarxa, podem
extreure que les direccions de la xarxa són 172.16.x.x, sent les x valors
entre 0 i 255, és a dir, l'adreça de xarxa és 172.16.0.0/16 i per tant
és de  \textbf{classe B}.
\subsection{Adreça de xarxa}
Com s'ha extret en l'anterior secció, la adreça de xarxa és 172.16.0.0.
\subsection{Adreça de broadcast}
Sabent l'adreça de xarxa, podem concloure que
l'adreça de broadcast és 172.16.255.255, ja que aquesta és l'última adreça
disponible de tota la xarxa, és a dir, la part del host de l'adreça a valor
actiu a tots els bits. Inclús amb aquesta informació, per confirmar que no
hi hagi hagut cap error, s'ha procedit a mirar l'adreça de broadcast en els
paquets tipus:\\
\begin{lstlisting}
	    !arp && eth.dst == ff:ff:ff:ff:ff:ff
\end{lstlisting}
Els paquets d'aquest tipus mostren com a direcció IP 172.16.255.255 per destí,
es pot confirmar la informació extreta en aquest apartat.
\subsection{Porta d'enllaç}
S'ha vist en la xarxa que s'empra el protocol DHCP, pel que, primerament
es busca aquels paquests que siguin DHCP ACK:\\
\begin{lstlisting}
		bootp.option.dhcp == 5
\end{lstlisting}
En aquest protocol i en aquest tipus de paquet, es pot trobar la informàció
referent al router,  dins de Bootstrap Protocol (ACK), en opcions de router.
En aquest camp s'especifíca que l'adreça és 172.16.20.1.
\section{Anàlisi de nivell de enllaç i xarxa}
\subsection{Protocols encapsulats en les trames de nivell 2}
Al llarg de tota la trama, es poden veure 2 protocols de nivell 2 de
xarxa, \textbf{Ethernet II} i \textbf{IEEE 802.3 Ethernet}. En les següents
subseccions s'explicarà el tipus d'encapsulament d'aquests
\subsubsection{Ethernet II}
Aquest tipus de trama s'utilitza en l'àmbit general, i es pot trobar en la majoria
de paquets de la captura. La seva estructura segueix la següent:\\
% Taula amb l'estructura
\subsubsection{IEEE 802.3 Ethernet}
Aquesta classe s'utilitza en els protocols de LLC. La seva estructuar és la
següent:
% Taula amb l'estructura
\subsection{Protocols encapsulats en trames de nivell 2}
Per a trobar els diferents protocols utilitzats, s'utilitza la eina de
\textit{Protocol Hierarchy}, accessible dins del menú d'estadístiques del
Wireshark. En aquest menú, podem veure com és divideix els protocols segons els
nivells, començant pel nivell físic, i seguint amb Ethernet. Dins d'aquest menú
es pot veure els següents tipus de paquets, que són: Logical-Link Control (LLC),
Internetwork Packet eXchange (IPX), Internet Protocol Version 6 (IPv6),
Internet Protocol Version 4 (IPv4), Address Resolution Protocol (ARP), que
s'explicaran a continuació, juntament amb el seu valor de tipus.\\
\begin{itemize}
\item ARP, amb valor 0x0806, s'encarrega de resoldre i mantenir de manera automàtoca
la taula d'equivalències entre les adreces MAC i les adreces IP dels nodes o màquines
que es comuniquen.
\item IPv4, amb valor 0x0800, és el protocol per excelència d'Internet. Serveix
per a la identiciació i connexió de nodes.
\item IPv6, amb valor 0x86dd, neix com a un protocol per a substituir IPv4, i
treure els problemes que sorgeixen amb aquest, com és la falta d'adreces, seguretat i
qualitat de servei. Moltes de les seves funcionalitats s'han portat enrere per al
protocol de IPv4.
\item IPX, amb valor 0x8137, s'utilitza per a transmetre datagrames entre els
diferents serviors i els programes de les estacions de treball.
\item LLC, sense valor donat que està encapsulat amb IEEE 802.3 Ethernet i aquest
no te nombre reservat pel tipus, defineix la forma en què les dades són transferides
sobre el medi físic, proporcionant servei a les capes superiors.
\end{itemize}

\section{Anàlisi nivell de transport}
\section{Conclusions}
\end{document}





































